\documentclass{beamer}
 
\usepackage[utf8]{inputenc}
\usepackage{ctex}
\usepackage{setspace}
\usepackage{float}
\usepackage{booktabs}
\usepackage{graphicx}
\usetheme{Madrid}
\usecolortheme{default}
 
%Information to be included in the title page:
\title{Neutron Dosimetry in BNCT}
\author{Zhongming Zhang}
\institute{Neuboron}
\date{2019.3.13}
 
 
 
\begin{document}


\frame{\titlepage}
\begin{frame}
\frametitle{Catalog}
\tableofcontents
\end{frame}



\begin{frame}
\section{Basic Theory}
\frametitle{Basic Theory}
\begin{center}
  \begin{spacing}{2}
    \begin{table}[H]
      \footnotesize
      \begin{tabular}{cc}
        \toprule  %添加表格头部粗线
        元素& 相\quad 互\quad 作\quad 用\\
        \midrule  %添加表格中横线
        氢& 弹性散射\quad 辐射俘获\quad H(n,gamma)D\\
        碳& 弹性散射\quad 非弹性散射\quad C(n,n'3$\alpha$)\quad C(n,n'$\alpha$)Be \\
        氮& 弹性散射\quad 非弹性散射\quad N(n,p)C\quad N(n,d)C等\\
        氧& 弹性散射\quad 非弹性散射\quad O(n,$\alpha$)C\quad O(n,p)N\\
        \bottomrule %添加表格底部粗线
        \end{tabular}
    \end{table}
  \end{spacing}
\end{center}
\end{frame}


\section{Calculation Methods}
\begin{frame}
\frametitle{Calculation Methods}
\framesubtitle{Dosimetry}
\begin{center}
  \Large
  $ Dose(j)= \int \phi(j,E)*Kerma(j,E)\,dE/V(j)  $
\end{center}
\end{frame}


\begin{frame}
\frametitle{Calculation Methods}
\framesubtitle{Neutron Flux in MCNP}
\begin{center}
\Large
  $ \phi(j,E) \approx \frac{S_0}{N} \sum_{k=1}^{N} \sum_{m}{w_m}^{(n)}(j,E){d_m}^{(n)}(j)$\\
  \quad \\
  $ {w_m}^{(n)}(j,E) = {{w}_{m-1}}^{(n)}(j,E)[ 1- \frac {\sum_{a}^{(m-1)}(j,E)}{\sum_{t}^{(m-1)}(j,E)}]  $
\end{center}
\end{frame}


\section{Algorithm}
\begin{frame}
\frametitle{Algorithm}
\framesubtitle{Basic flow}
%\setlength{\parindent}{4em}
\begin{spacing}{1.5}
 1.从CT读取图像文件,获得数据。\\
 2.对数据文件进行处理,划分网格(均匀或非均匀)\\
 3.计算网格中心点坐标\\
 4.通过CT文件可以得知网格中心点的材料与密度,利用中心点方法快速确定该网格的材料与密度。\\
 5.生成MCNP 或 蒙卡输入文件。\\
 6.获得相应网格的中子通量或中子通量率以及中子能量谱,通过读取数据库的Kerma factors进行剂量计算。\\
 7.通过剂量生成相应图表以及图像。
\end{spacing}
\end{frame}

\begin{frame}
\frametitle{Algorithm}
\framesubtitle{Particle track Algorithm in MCNP}
\begin{figure}[H]%%图
	\centering  %插入的图片居中表示
	\includegraphics[width=0.7\textwidth]{mcnp.jpg}  %插入的图,包括JPG,PNG,PDF,EPS等,放在源文件目录下
	\caption{Particle track in MCNP}  %图片的名称
	%\label{fig:mcmthesis-logo}   %标签,用作引用
\end{figure}
\end{frame}


\begin{frame}
\frametitle{Algorithm}
\framesubtitle{Fast particle track method}
\begin{figure}[H]%%图
	\centering  %插入的图片居中表示
	\includegraphics[width=0.7\textwidth]{track.jpg}  %插入的图,包括JPG,PNG,PDF,EPS等,放在源文件目录下
	\caption{Fast particle track}  %图片的名称
	%\label{fig:mcmthesis-logo}   %标签,用作引用
\end{figure}
\end{frame}
  

\begin{frame}
\section{Challenge}
\frametitle{Challenge}
\setlength{\parindent}{4em}
\begin{spacing}{1.5}
  1.硼药的浓度变化导致必须加快蒙卡计算速度\\
  2.混合网格的快速射线轨迹追踪\\
\end{spacing}
\end{frame}


\begin{frame}
\begin{center}
\Huge Thank you !
\end{center}
\end{frame}
 

\end{document}